\documentclass[12pt,a4paper]{article}
\usepackage{times}
\usepackage{durhampaper}
\usepackage{harvard}

\citationmode{abbr}
\bibliographystyle{agsm}

\title{Inverse Uncertainty Quantification Methods for Numerical Storm Surge Models}
\author{} % leave; your name goes into \student{}
\student{Benjamin Russell}
\supervisor{Tobias Weinzierl}
\degree{MEng Computer Science}

\date{}

\begin{document}

\maketitle

\begin{abstract}
	
{\bf Context/Background} - 	When modelling storm surges using numerical methods, it is important that the model is correctly calibrated to ensure accurate results. Doing this without empirical measurements of the site being simulated is a hard and time consuming problem, and for this reason there exist many different methods for estimating correct model parameters.

{\bf Aims} - This paper aims to compare a variety of different techniques for uncertainty quantification to infer correct parameters. This will allow a comparison between the time taken to estimate the parameters and their accuracy when compared with real data for storm surges. This comparison will highlight which techniques are more appropriate in real world scenarios.

{\bf Method} - The methodology used in this paper was to implement three different techniques one of which is a hybrid of the other two. These were then used to infer and calibrate the model parameter values and the results were then tested in the model on a variety of real world storm surges with their accuracies and runtimes compared.
\end{abstract}

\begin{keywords}
Storm surge, uncertainty quantification, inverse problem, Bayesian inference, Monte-Carlo, Surrogate Model
\end{keywords}

\section{Introduction}
When severe storms hit coastal regions, the high winds and atmospheric pressure can cause a sea level rise. This affect is known as a storm surge. Storm surges pose a grave danger to coastal communities both from an economic and a human perspective. From 1963 to 2012 $49\%$ of all deaths caused by hurricanes in the US were due to the subsequent storm surge (REF DEATH PAPER). It has also been estimated that if coastal flood defences are not improved in US coastal regions then by 2050 then yearly losses could exceed US\$1trillion (REF COST PAPER).  To address this issue it is necessary to plan and implement better coastal flood defences. One such way to plan this new infrastructure is to use numerical modelling techniques to simulate hypothetical storm surge events, induced by hurricanes, and study the simulated sea level rise to assess coastal regions most at risk to loss of human life and loss of property. This poses a problem as storm surge models are very complex and their accuracy depends inherently on how well calibrated the input parameters are. To address this, without having accurate measurements of the underlying phenomena governing these the parameters, it is necessary to estimate them and calibrate the model accordingly. Unfortunately due to the complex nature of the numerical models this is time-prohibited and therefore more sophisticated methods of estimation and calibration are used to infer these parameters and quantify the uncertainty of them.

In this paper we focus on using a range of different techniques that fall under the umbrella of surrogate modelling. This is the use of less accurate models that are less computationally expensive to run then the full model (REF QUIRANTE). The remainder of this section will give an overview of storm surge modelling, inverse uncertainty quantification, surrogate modelling, and model calibration as well as the motivations for comparing the various techniques.

\subsection{Background}

\section{Related Work}
This section presents a survey of existing work on the problems that this project addresses.  it should be between 2 to 4 pages in length.  The rest of this section shows the formats of subsections as well as some general formatting information for tables, figures, references and equations.

\subsection{Main Text}

The font used for the main text should be Times New Roman (Times) and the font size should be 12.  The first line of all paragraphs should be indented by 0.25in, except for the first paragraph of each section, subsection, subsubsection etc. (the paragraph immediately after the header) where no indentation is needed.

\subsection{Figures and Tables}
In general, figures and tables should not appear before they are cited.  Place figure captions below the figures; place table titles above the tables.  If your figure has two parts, for example, include the labels ``(a)'' and ``(b)'' as part of the artwork.  Please verify that figures and tables you mention in the text actually exist.  make sure that all tables and figures are numbered as shown in Table \ref{units} and Figure 1.
%sort out your own preferred means of inserting figures

\begin{table}[htb]
\centering
\caption{UNITS FOR MAGNETIC PROPERTIES}
\vspace*{6pt}
\label{units}
\begin{tabular}{ccc}\hline\hline
Symbol & Quantity & Conversion from Gaussian \\ \hline
\end{tabular}
\end{table}

\subsection{References}

The list of cited references should appear at the end of the report, ordered alphabetically by the surnames of the first authors.  References cited in the main text should use Harvard (author, date) format.  When citing a section in a book, please give the relevant page numbers, as in \cite[p293]{budgen}.  When citing, where there are either one or two authors, use the names, but if there are more than two, give the first one and use ``et al.'' as in  , except where this would be ambiguous, in which case use all author names.

You need to give all authors' names in each reference.  Do not use ``et al.'' unless there are more than five authors.  Papers that have not been published should be cited as ``unpublished'' \cite{euther}.  Papers that have been submitted or accepted for publication should be cited as ``submitted for publication'' as in \cite{futher} .  You can also cite using just the year when the author's name appears in the text, as in ``but according to Futher \citeyear{futher}, we \dots''.  Where an authors has more than one publication in a year, add `a', `b' etc. after the year.

\section{Solution}

This section presents the solutions to the problems in detail.  The design and implementation details should all be placed in this section.  You may create a number of subsections, each focussing on one issue.  

This section should be between 4 to 7 pages in length.

\section{Results}

this section presents the results of the solutions.  It should include information on experimental settings.  The results should demonstrate the claimed benefits/disadvantages of the proposed solutions.

This section should be between 2 to 3 pages in length.

\section{Evaluation}

This section should between 1 to 2 pages in length.

\section{Conclusions}

This section summarises the main points of this paper.  Do not replicate the abstract as the conclusion.  A conclusion might elaborate on the importance of the work or suggest applications and extensions.  This section should be no more than 1 page in length.

The page lengths given for each section are indicative and will vary from project to project but should not exceed the upper limit.  A summary is shown in Table \ref{summary}.

\begin{table}[htb]
\centering
\caption{SUMMARY OF PAGE LENGTHS FOR SECTIONS}
\vspace*{6pt}
\label{summary}
\begin{tabular}{|ll|c|} \hline
& \multicolumn{1}{c|}{\bf Section} & {\bf Number of Pages} \\ \hline
I. & Introduction & 2--3 \\ \hline
II. & Related Work & 2--3 \\ \hline
III. & Solution & 4--7 \\ \hline
IV. & Results & 2--3 \\ \hline
V. & Evaluation & 1-2 \\ \hline
VI. & Conclusions & 1 \\ \hline
\end{tabular}
\end{table}


\bibliography{projectpaper}


\end{document}